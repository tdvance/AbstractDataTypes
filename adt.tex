\documentclass{amsart}

\title{Abstract Data Types}
\author{Todd D. Vance}
\date{\today}

\begin{document}
\maketitle{}

\section{Local Graph}

A local graph (modeling a directed graph, loops and multiple edges allowed, from which only a node and its immediate neighborhood are visible at any one time) is actually a system of ADTs.  It consists of a single Cursor, and a finite number of Nodes and outgoing Edges indexed by Directions from a node.  In addition, there is a concept of Items. An Item belongs to something, either a Node, another Item, the Cursor, or it belongs to nothing (the Nil object). The motivation for Items is Zork-like text adventure games, in which a Node is a room, the Cursor is the player, and Directions are connections to other rooms.  Items can then be in a room, on the player, in or on another item, or nowhere (for a period of time).

The axioms are as folows:

\begin{enumerate}
\item There exists a unique Cursor object $c$.
\item There exists a unique Nil object $\epsilon$.
\item There exists a unique Node object $n$ such that $c\in{n}$.
\item For each Direction $d$ and Node object $n$, either $n.d=\epsilon$ or there exists a unique Node object $m$ such that $n.d=m$.
\item For each Item object $i$, exactly one of the following hold:
\begin{enumerate}
\item $i\in\epsilon$.
\item $i\in{c}$.
\item There exists a unique Node object $n$ such that $i\in{n}$
\item There exists a unique Item object $j$ such that $i\in{j}$
\end{enumerate}
\item There are no circular inclusions among items; that is for no sequence of Item objects $i_1, i_2, \dots, i_n$ does it happen that $i_1\in i_2 \in \cdots\in i_n\in i_1$.
\end{enumerate}

For practical purposes, some query operations are permitted:

\begin{enumerate}
\item If $n$ is a Node object, then $\mathrm{dir}(n)$ is a set containing all directions  $d$ such that $n.d\ne\epsilon$.
\item $\mathrm{loc}(c)$ is the unique Node object $n$ such that $n\in{c}$.
\item $\mathrm{loc}(i)=o$ is equal to the object $o$ which must be exactly one of the following:
\begin{enumerate}
\item $o=\epsilon$ if $i\in\epsilon$.
\item $o=n$ if $n$ is a Node object and $i\in{n}$.
\item $o=j$ if  $j$ is an Item object and $i\in{j}$.
\item $o=c$ if $i\in{c}$.
\end{enumerate}
\end{enumerate}

Optional operations provide more functionality:
\begin{enumerate}
\item If $n$ is a Node object, then $itm(n)$ is the set of all Item objects $i$ such that $i\in{n}$.
\item If $i$ is an Item object, then $itm(i)$ is the set of all Item objects $j$ such that $j\in{i}$
\item $itm(c)$ is the set of Item objects $i$ such that $i\in{c}$.
\item $itm(\epsilon)$ is the set of Item objects $i$ such that $i\in\epsilon$.
\item $itm()$ is the set of all Item objects.
\item $nod()$ is the set of all Node objects.
\end{enumerate}
\end{document}
