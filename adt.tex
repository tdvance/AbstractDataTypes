\documentclass{amsart}

\title{Abstract Data Types}
\author{Todd D. Vance}
\date{\today}

\begin{document}
\maketitle{}

\section{Abstract Data Type}
One method for designing classes in an object-oriented fashion is via an \emph{abstract data type}.  The various abstract data types can be defined formally or semi-formally, or in a casual language.  There are benefits to each of the various levels of formality.  The author proposes an axiomatic approach (similar to Nell/Walker (http://dl.acm.org/citation.cfm?id=230973), though perhaps less formally; see also Liskov/Zilles 1974, Programming with Abstract Data Types) as a way of accurately communicating the specification of a class as well as unit tests.  This approach to abstract data types can be thought of as being to the design of a system as ``test driven development'' is to the writing of the programs for a system.


\section{Set}
The ADT $\mathbf{set}$ can be specified by the following operations and axioms.  We assume another type $\mathbf{E}$ representing the type of the elements of the set.  This specifies an immutable set, where elements are added or removed by returning a new set.  Here are the supported operations:

\begin{enumerate}
\item $\mathrm{empty}:\emptyset\rightarrow\mathbf{set}$.
\item $\mathrm{size}:\mathbf{set}\rightarrow\mathbf{N}$ (where $\mathbf{N}$ represents nonnegative integers).
\item $\mathrm{contains}:\mathbf{set}\times\mathbf{E}\rightarrow\mathbf{bool}$ (where $\mathbf{bool}=\{\mathrm{true},\mathrm{false}\}$ represents the Boolean type).
\item $\mathrm{add}:\mathbf{set}\times\mathbf{E}\rightarrow\mathbf{set}$
\item $\mathrm{remove}:\mathbf{set}\times\mathbf{E}\rightarrow\mathbf{set}$
\end{enumerate}

Informally, $\mathrm{empy}$ is a nullary operation, that is, a constant.  It represents the empty set.  The $\mathrm{size}$ operation applied to any set returns the number of elements.  The $\mathrm{contains}$ operation applied to a set and an element returns $\mathrm{true}$ if the element is in the set.  Otherwise, it returns $\mathrm{false}$.  The $\mathrm{add}$ operation applied to a set and an element returns the set whose elements are the elements of the specified set along with the specified element.  The $\mathrm{remove}$ operation applied to a set and an element returns the set whose elements are the elements of the specified set except for the specified element.

The axioms specifying the behavior of the operations are as follows:

\begin{enumerate}
\item $\mathrm{size}(\mathrm{empty})=0$
\item if $\mathrm{contains}(s, e)=\mathrm{false}$, then $\mathrm{size}(\mathrm{add}(s, e)) = \mathrm{size}(s) + 1$
\item if $\mathrm{contains}(s, e)=\mathrm{true}$, then $\mathrm{add}(s, e) = s$
\item if $\mathrm{contains}(s, e)=\mathrm{true}$, then $\mathrm{size}(\mathrm{remove}(s, e)) = \mathrm{size}(s) - 1$
\item if $\mathrm{contains}(s, e)=\mathrm{false}$, then $\mathrm{remove}(s, e) = s$
\item $\mathrm{contains}(empty, e) = \mathrm{false}$
\item $\mathrm{contains}(\mathrm{add}(s, e), e)=\mathrm{true}$
\item If $e\ne{f}$, then $\mathrm{contains}(\mathrm{add}(s, e), f)=\mathrm{contains}(s,f)$
\item $\mathrm{contains}(\mathrm{remove}(s, e), e)=\mathrm{false}$
\item If $e\ne{f}$, then $\mathrm{contains}(\mathrm{remove}(s, e), f)=\mathrm{contains}(s,f)$
\end{enumerate}


One could specify a mutable version of $\mathbf{set}$ as a structure having just one modifiable attribute: the immutable set.  Thus, we have something like $s$, such that $s.\mathrm{set}$ is a $\mathbf{set}$.  We can modify $s$: $s.\mathrm{set} \leftarrow t$ where $t$ is some $\mathbf{set}$.

Instead of $\mathrm{empty}$ is $\mathrm{new~\mathbf{set}}$ which is a structure whose attribute is equal to $\mathrm{empty}$.    So, $s\leftarrow \mathrm{new~\mathbf{set}}$ causes $s.\mathrm{set}$ to equal $\mathrm{empty}$.

The $\mathrm{add}$ operation then modifies $s$ as follows: $\mathrm{add}(s, e)$ performs the operation $s.\mathrm{set}\leftarrow \mathrm{add}(\mathrm{set}, e)$.

Similarly, the $\mathrm{remove}$ operation modifies $s$ as follows: $\mathrm{remove}(s, e)$ performs the operation $s.\mathrm{set}\leftarrow \mathrm{remove}(\mathrm{set}, e)$.

The remaining operations just operate directly on $s.\mathrm{set}$.  So, $\mathrm{contains}(s, e)$ is equal to $\mathrm{contains}(s.\mathrm{set}, e)$, and likewise for the other operations.

The analogous scheme can be used to turn any immutable ADT having operations that return different instances of the ADT into a corresponding mutable ADT.



\end{document}
